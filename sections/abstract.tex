\newpage

\section*{ABSTRACT}
\addcontentsline{toc}{section}{ABSTRACT}

Image Super Resolution (ISR) techniques have been in development for a long time. It has witnessed substantial advancements in recent years, aiming to enhance the visual quality of low-resolution images. This study compares different ISR technique based on deep learning and suggest which works best on different scenarios. This paper includes comparison of SISR techniques including SRGAN, SRResNet and SRCNN. Each technique has distinct architectural approaches and training methodologies, addressing the common goal of reconstructing high-resolution images from their low-resolution counterparts. PSNR and SSIM are used as evaluation metrics. 29.83 PSNR and 0.887 SSIM for Set5 with SRResNet and 26.132 PSNR and 0.821 SSIM for Set 5 with SRGAN were achieved in this study. The study aims to provide valuable insights for enthusiasts and researcher interested in using SISR techniques for high-quality image super-resolution. Each of these techniques has its strengths and weaknesses, and understanding them can help researchers and enthusiasts choose the most suitable method for their specific needs.
\\
\\
\textit {{\bf Keywords:}  Super Resolution Generative Adversarial Network (SRGAN), Super Resolution Residual Network (SRResNet), Super Resolution Convolutional Neural Network (SRCNN), Single Image Super Resolution (SISR) }