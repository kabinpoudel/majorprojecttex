\newpage
\pagenumbering{arabic}
\section{INTRODUCTION}
Image super resolution is a technique used to enhance the resolution and quality of
image beyond its original size. It is useful technique in the field of image processing. It
can be used to enhance the quality of image captured in night, shaky image etc. There
are various techniques and algorithm used for image super resolution, including both
traditional methods and deep learning-based approaches. Traditional super image
resolution techniques are based on mathematics techniques. These techniques have
been used for many years. Some of the traditional super resolution techniques include:
Interpolation, Edge-Directed interpolation, Bayesian methods etc. Traditional methods
have certain limitations compared to deep learning approaches. They may struggle to
capture complex patterns and textures, and their performance is often constrained by
the hand-crafted features and assumptions used in the algorithms.
Deep learning based super resolution typically employs Convolution Neural
Network (CNN) to learn the mapping between low-resolution and high-resolution
images. The network is trained on large dataset of paired low-resolution and high-
resolution images. During training, the networks learn to identify patterns and features
that enables it to generate high-resolution details from low-resolution inputs.

% \clearpage

\subsection{Statement of problem}
The need of super image resolution arises when we encounter low-resolution images
that lack the desired level of detail and clarity that may be due to various factors such as limitations in capturing hardware, resizing
operations, or compression from social media. There is not only need of Super Resolution, We also need to know which method works best for which usecase.
% add some lines of  content and remove newpage  below

\subsection{Objectives} 
% spandan
\begin{itemize}
    \item To implement different Super Resolution Algorithms.
    \item To compare them to findout their best usecases.
\end{itemize}

\subsection{Scope and Applications}
Image super resolution has numerous practical applications in field
like photography, surveillance, and digital art where enhanced image quality is crucial
for accurate decision-making, analysis, and interpretation. The development of efficient
and accurate super-resolution method involves exploring traditional signal processing
techniques, as well as cutting-edge deep learning approaches, to achieve optimal results
while considering computational efficiency and resource constraints. We can understand that not every technique will work well in every scenarios. So, There is a need for understanding the techniques to use them where they perform best.\\
Super image resolution can be used for following things: 
\begin{enumerate}
    \item Super resolution can be used to improve the resolution of old and degraded
    images, preserving historical photographs, artworks, and documents. 
    \item Super resolution can be used in digital cameras and smartphones to provide
    better zoom capabilities without significant loss of image quality.
    \item Super resolution can enhance the resolution of individual frames in videos,
    leading to improved video quality and better extraction of information. 
    \item In surveillance systems, low resolution camera feeds can be upscaled to improve
    the ability to identify faces, license plates, or other critical details, helping in
    investigations. 
    
\end{enumerate}
