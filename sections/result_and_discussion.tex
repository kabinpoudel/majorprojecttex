\newpage
\section{RESULT AND DISCUSSION}
\subsection{Current Result}
% Table
\begin{table}[h]
    \centering
    \begin{tabular}{|c|c|c|c|c|c|c|}
    \hline
    &PSNR & SISM & PSNR & SISM & PSNR & SISM \\
    \hline
    ResNet&29.83 & 0.887 & 28.275 & 0.795 & 27.203 & 0.751 \\
    \hline
    SRGAN &23.946 & 0.821 &25.112 & 0.715 & 24.631 & 0.662 \\
    \hline
    & \multicolumn{2}{|c|}{Set5}& \multicolumn{2}{|c|}{Set14} & \multicolumn{2}{|c|}{BSD100} \\
    \hline
    \end{tabular}
    \caption{Evaluation Metrics in Test Datasets}
\end{table}

\subsection{Work Completed:}
As we are planning on comparison of different methods. We have selecteed few methods like SRCNN, SRGAN, and ESRGAN. SRGAN involves residual network SRResNet which will be also used as baseline for SRGAN. We have trained and tested SRGAN on various sets of data involving CelebA, DIV2k, Flickr2k, OST and COCO(2014). CoCo has a large and diverse set of data and it perfromed well thats why we are finalizing this dataset for training further methods. 
\subsection{Work Remaining}
\begin{enumerate}
    \item We have not trained models for SRCNN and ESRGAN yet.
    \item We don't have a polished website to showcase our results.
\end{enumerate}  
\subsection{Limitations}
Few limitations related to SRGAN that we have trained are:
\begin{enumerate}
    \item The implementation and training of SRGAN demand substantial computational resources, posing challenges in terms of the time and hardware needed.
    \item  The effectiveness of SRGAN relies heavily on the availability of a sizable and diverse training dataset, presenting challenges in obtaining and curating such data.
    \item There is alot of subjectivity in evaluating the images as quantative metrics can't identify human perception.
    \item  The study lacks comparison with real-world benchmarks or scenarios.
    \item Due to constraints, we couln't explore tinkering hyperparameters.
    \item The study lacks a direct comparison with the latest state-of-the-art super-resolution techniques.
\end{enumerate}
% \subsection{Problem Faced}
% \subsection{Budget Analysis}
\subsection{Work Schedule:}
\newcolumntype{P}[1]{>{\raggedright\vrule height4ex width 0pt}p{#1}<{\vrule depth 2.5ex width 0pt}}
    \begin{tabular}{|P{6cm}*{10}{|c}|}
    \hline
    \centering \raisebox{-2ex}[0pt][0pt]{Action plan} & \multicolumn{7}{c|}{2023} & \multicolumn{3}{c|}{2024} \\
    \cline{2-11}
    \multicolumn{1}{|c|}{\vphantom{$\Big|$}} &
    \scriptsize Jun & \scriptsize Jul & \scriptsize Aug & \scriptsize Sep &
    \scriptsize Oct & \scriptsize Nov & \scriptsize Dec & \scriptsize Jan &
    \scriptsize Feb & \scriptsize Mar \\
    \hline
    Research &
    {\cellcolor{gray}} &{\cellcolor{gray}}&{\cellcolor{gray}}&{\cellcolor{gray}}&{\cellcolor{gray}}&&&&& \\
    \hline
    Data Acquisition &
    && {\cellcolor{gray}} &{\cellcolor{gray}} &{\cellcolor{gray}} &&&&& \\
    \hline
    Model Development and Evaluation &
    &&&&{\cellcolor{gray}} &{\cellcolor{gray}}&{\cellcolor{gray}}&&& \\
    \hline
    Model Deployment &
    &&&&&&{\cellcolor{gray}}&{\cellcolor{gray}}&{\cellcolor{gray}} & \\
    \hline
    Prepare the report and presentation &
    &&&&&&{\cellcolor{gray}}&{\cellcolor{gray}}&{\cellcolor{gray}}&{\cellcolor{gray}} \\
    \hline
    \end{tabular}