\newpage
\section{LITERATURE REVIEW}
In the timeline of image scaling process, the first used methods were different interpolation techniques and sparse representation-based methods. But with the advent
of deep-learning based image scaling techniques interpolation methods were less commonly used in image enhancement.

D. Han (2013) {\bf“Comparison of commonly used image interpolation methods”} discusses multiple interpolation methods like nearest neighbor, bilinear, bicubic, etc. Although interpolation is fast but have some disadvantages. But using these methods was not optimal for our use case as they may lead to the loss of fine details and sharpness in image and can amplify noise present in image which may not be accurate in real-world images.\cite{r1}

C. Dong, et. al, {\bf“Image super-resolution using deep convolutional networks”} is a seminal paper on use of Deep Learning for the use of for the task of SR. It only consists of three layers and requires the LR image to be up-sampled using bicubic interpolation prior to being processed by the network, but it was shown to outperform the state-ofthe-art methods of that time.\cite{r2}

C. Ledig et al., {\bf“Photo-realistic single image super-resolution using a generative adversarial network”} discusses about a Generative Adversarial Network (GAN) for image super-resolution (SR). It is the first framework capable of inferring photo-realistic natural images for 4x upscaling factors using GAN. To achieve this, it uses
a perceptual loss function which consists of an adversarial loss and a content loss. The adversarial loss pushes the solution to the natural image manifold using a discriminator
network that is trained to differentiate between the super-resolved images and original photo-realistic images. In addition, it uses a content loss motivated by perceptual
similarity in VGG space instead of similarity in pixel RGB space. This deep residual network is able to recover photo-realistic textures from heavily down sampled images on public
benchmarks. An extensive mean-opinion-score (MOS) test shows hugely significant gains in perceptual quality using SRGAN. The MOS scores obtained with SRGAN are
closer to those of the original high-resolution images than to those obtained with any state-of-the-art method at the time. Both SRResNet and SRGAN are explained in related theory portion of the study.\cite{r3}
